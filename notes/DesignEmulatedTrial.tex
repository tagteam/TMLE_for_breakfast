% Created 2023-05-20 Sat 10:42
% Intended LaTeX compiler: pdflatex
\documentclass{beamer}\usepackage{listings}
\usepackage{color}
\usepackage{amsmath}
\usepackage{array}
\usepackage[T1]{fontenc}
\usepackage{natbib}
\lstset{
keywordstyle=\color{blue},
commentstyle=\color{red},stringstyle=\color[rgb]{0,.5,0},
literate={~}{$\sim$}{1},
basicstyle=\ttfamily\small,
columns=fullflexible,
breaklines=true,
breakatwhitespace=false,
numbers=left,
numberstyle=\ttfamily\tiny\color{gray},
stepnumber=1,
numbersep=10pt,
backgroundcolor=\color{white},
tabsize=4,
keepspaces=true,
showspaces=false,
showstringspaces=false,
xleftmargin=.23in,
frame=single,
basewidth={0.5em,0.4em},
}
\institute{Section of Biostatistics, University of Copenhagen}
\subtitle{}
\RequirePackage[absolute, overlay]{textpos}
\setbeamertemplate{footline}[frame number]
\setbeamertemplate{navigation symbols}{}
\RequirePackage{fancyvrb}
\RequirePackage{array}
\RequirePackage{multirow}
\RequirePackage{tcolorbox}
\definecolor{mygray}{rgb}{.95, 0.95, 0.95}
\newcommand{\mybox}[1]{\vspace{.5em}\begin{tcolorbox}[boxrule=0pt,colback=mygray] #1 \end{tcolorbox}}
\newcommand{\sfootnote}[1]{\renewcommand{\thefootnote}{\fnsymbol{footnote}}\footnote{#1}\setcounter{footnote}{0}\renewcommand{\thefootnote}{\arabic{foot note}}}
\usepackage{alphalph}
\alphalph{\value{footnote}}
\setbeamertemplate{itemize item}{\textbullet}
\setbeamertemplate{itemize subitem}{-}
\setbeamertemplate{itemize subsubitem}{}
\setbeamertemplate{enumerate item}{\insertenumlabel.}
\setbeamertemplate{enumerate subitem}{\insertenumlabel.\insertsubenumlabel}
\setbeamertemplate{enumerate subsubitem}{\insertenumlabel.\insertsubenumlabel.\insertsubsubenumlabel}
\setbeamertemplate{enumerate mini template}{\insertenumlabel}
\makeatletter\def\blfootnote{\xdef\@thefnmark{}\@footnotetext}\makeatother
\newcommand{\E}{\ensuremath{\mathrm{E}}}
\renewcommand{\P}{\ensuremath{\mathrm{P}}}
\renewcommand{\d}{\ensuremath{\mathrm{d}}}
\setbeamertemplate{caption}{\raggedright\insertcaption\par}

\renewcommand*\familydefault{\sfdefault}
\itemsep2pt
\usepackage[utf8]{inputenc}
\usepackage[T1]{fontenc}
\usepackage{graphicx}
\usepackage{longtable}
\usepackage{wrapfig}
\usepackage{rotating}
\usepackage[normalem]{ulem}
\usepackage{amsmath}
\usepackage{amssymb}
\usepackage{capt-of}
\usepackage{hyperref}
\usetheme{default}
\author{Thomas Alexander Gerds}
\date{May 22, 2023}
\title{Design of an emulated trial using modern causal inference}
\begin{document}

\maketitle

\section{Introduction}
\label{sec:org4c9867f}

\begin{frame}[label={sec:org0abe3ef}]{The overall idea}
In causal inference, an emulated target trial is a
method for estimating the effects of a treatment or
intervention.\footnote{Hernan and Robins. Am J Epi, 183(8):758--764, 2016.} \vfill

To conduct an emulated target trial, researchers use observational
data and statistical methods to construct a hypothetical trial that
emulates the conditions of a randomized controlled trial. 
\vfill

This involves selecting a study population that is similar to a
population that would have been eligible for the hypothetical
randomized trial and modeling the treatment assignment process.
\vfill

The researchers then simulate the outcomes that would have been
observed if the hypothetical trial had been conducted.
\end{frame}

\begin{frame}[label={sec:orgffc08de}]{(Limitations of) Register data}
The most important challenges that have to be resolved are:
\vfill

\begin{enumerate}
\item how to identify \alert{when} a person is eliglible?
\item identify variables that were used (in the real world) to decide whether the person initiated treatment
\item collect data to determine adherence to treatment during followup
\item collect variables that were used to decide treatment changes
\item collect predictors for outcome(s), censoring, and competing risks
\item force all this information into a \alert{discrete time grid}
\end{enumerate}
\end{frame}

\begin{frame}[label={sec:org3083828}]{The statistical analysis plan}
\begin{block}{Design}
\begin{enumerate}
\item Inclusion/Exclusion \(\rightarrow\) time zero
\item Width of time grid intervals
\item Treatment regimens (static, stochastic, dynamic)
\item Target parameters (estimands, contrasts)
\end{enumerate}
\end{block}

\begin{block}{Analysis}
\begin{itemize}
\item Specification of nuisance parameter estimation, history adjusted:
\begin{itemize}
\item propensity score models
\item censoring models
\item outcome models
\end{itemize}
\item List of learners for super learning (pre-specified ands much more
flexible than a single regression model)
\end{itemize}
\end{block}
\end{frame}

\section{From target population to study population}
\label{sec:orgf2d9620}

\setbeamercolor{background canvas}{bg=black}

\begin{frame}[label={sec:orgd56e057}]{}
\huge \color{white}

When is time zero?
\end{frame}

\setbeamercolor{background canvas}{bg=white}  

\begin{frame}[label={sec:org8a81169}]{When is time zero}
In the trials actually designed and carried out in the \alert{real world}, a
prespecified trial protocol defines the target population and how a
study population would be recruited from eligible patients, and
therefore the definition of time zero of any participant already in
the study is obvious.  \vfill

However, during the recruitment process of a \alert{hypothetical
trial}, there always exist multiple possibilities in defining a study
population and the associated time zeros.
\end{frame}

\begin{frame}[label={sec:orgd06534d}]{When is time zero: register data}
\begin{description}
\item[{Strategy 1}] include everyone in the register who passes some given
eligibility criteria:
\begin{description}
\item[{a}] set time zero at the first pass and look back in time to define treatment status
\item[{b}] grace period: set time zero x days after date of first pass. exlude previously treated. 
treatment initiation happens in grace period
\end{description}
\end{description}
\vfill\pause     
\begin{description}
\item[{Strategy 2}] include everyone in the register who starts 
the treatment:
\begin{description}
\item[{a}] compare with patients who start a different treatment
\item[{a}] match with other patients who could have started the treatment but did not (not yet)
\end{description}
\end{description}
\end{frame}

\begin{frame}[label={sec:org7923855}]{When is time zero: illustration}
\begin{block}{Example inclusion}
Include all outcome-free patients above 50 who have a 5 year or longer history of
diabetes treatment when they have a recent HbA1c measurement above 58
mmol/mol.  \vfill\pause
\end{block}

\begin{block}{Challenges}
\begin{itemize}
\item not all patients get their HbA1c measured regularly
\item the register data do not contain all HbA1c measurements
\item there can be HbA1c above 58 during the first 5 diabetes years
\item a single patient can fulfill the eligibility criteria multiple times
according to the registered data\footnote{A real randomized trial would not include the same patient multiple times}\textsuperscript{,}\,\footnote{Subset analyses (e.g., age > 60 or HbA1c > 65) can potentially
include a patient who is not in this subgroup for the main analysis at
a different date.}
\end{itemize}
\end{block}
\end{frame}
\setbeamercolor{background canvas}{bg=black}

\begin{frame}[label={sec:orgb7f83fd}]{}
\huge \color{white}

Treatment regimens of the hypothetical target trial
\end{frame}

\setbeamercolor{background canvas}{bg=white}

\begin{frame}[label={sec:org7eea4ec}]{Treatment regimens}
\begin{block}{Static treatment regimen}
A fixed treatment strategy that does not change based on individual
patient responses/characteristics.
\end{block}

\begin{block}{Dynamic treatment regimen}
A personalized treatment strategy that adapts treatment decisions
based on individual patient responses over time. Used in situations
where patients require sequential or multiple treatments, and the
optimal treatment may depend on how the patient responded to previous
treatments.
\end{block}

\begin{block}{Stochastic treatment regimen}
Assigns treatments to patients with varying probabilities based on a
probabilistic algorithm. Stochastic treatment regimens are useful when
the optimal treatment is uncertain, and there are multiple treatments
with similar expected outcomes.
\end{block}
\end{frame}

\begin{frame}[label={sec:orgbc0815b},fragile]{Examples}
 \begin{block}{Static treatment regimens}
\begin{itemize}
\item Patients continue treatment A for x years
\item Patients continue to not use treatment A for x years (placebo is not an option!)
\item Patients continue treatment A for x years but are not allowed to use treatment B
\end{itemize}
\end{block}

\begin{block}{Dynamic treatment regimens}
\begin{itemize}
\item If measured \texttt{HbA1c > 60} then intensify dose of treatment A or add treatment B
\item If patient has a diagnosis with myocardiac infarction then add treatment B
\end{itemize}
\end{block}

\begin{block}{Stochastic treatment regimens}
\begin{itemize}
\item Patients add treatment B with a given probability conditional on
patient characteristics, treatment history and comorbidity
\end{itemize}
\end{block}
\end{frame}

\section{Outcomes}
\label{sec:org4676376}
\setbeamercolor{background canvas}{bg=black}

\begin{frame}[label={sec:orga004107}]{}
\huge \color{white}

Target parameters
\end{frame}

\setbeamercolor{background canvas}{bg=white}  

\begin{frame}[label={sec:org7a71644}]{Target parameters}
For one treatment regimen, the expected outcome t-years after time
zero.
\vfill

For two or multiple treatment regimens, target causal effects (estimands) include

\begin{itemize}
\item Average Treatment Effect (ATE): The average
difference in outcomes\footnote{Competing risks affect the interpretation!} t-years after time zero
between two treatment regimens in the eligible patients
\item Average Treatment Effect on the Treated (ATT): same as ATE but in
individuals who received the treatment
\end{itemize}
\end{frame}
\end{document}